% last updated in April 2002 by Antje Endemann
% Based on CVPR 07 and LNCS, with modifications by DAF, AZ and elle, 2008 and AA, 2010, and CC, 2011; TT, 2014; AAS, 2016

\documentclass[runningheads]{llncs}
\usepackage{graphicx}
\usepackage{amsmath,amssymb} % define this before the line numbering.
\usepackage{ruler}
\usepackage{color}
\usepackage[width=122mm,left=12mm,paperwidth=146mm,height=193mm,top=12mm,paperheight=217mm]{geometry}
\begin{document}
% \renewcommand\thelinenumber{\color[rgb]{0.2,0.5,0.8}\normalfont\sffamily\scriptsize\arabic{linenumber}\color[rgb]{0,0,0}}
% \renewcommand\makeLineNumber {\hss\thelinenumber\ \hspace{6mm} \rlap{\hskip\textwidth\ \hspace{6.5mm}\thelinenumber}}
% \linenumbers
\pagestyle{headings}
\mainmatter
\def\ECCV18SubNumber{31}  % Insert your submission number here

\title{Self-Driving Assistant in Computer Simulation Environment} % Replace with your title

\titlerunning{ENGN4528 Group \ECCV18SubNumber}

\authorrunning{ENGN4528 Group \ECCV18SubNumber}

\author{Zhiyuan Chen, Xingyuan Xu, Qiusi Xiang, Bisyri Hisham, Kavinenh Mohanraj}
\institute{Australian National University}


\maketitle

\begin{abstract}
This project implement a self-driving assistant in computer 
simulation environment with Spatial CNN and Mask RCNN on a 
distributed system.
\dots
\keywords Self-Driving, Spatial CNN, Mask RCNN, Distributed 
System
\end{abstract}


\section{Introduction}
Self-driving technologies had been used in airplanes (known 
as AP for Auto Pilot), and trains (known as ATO for Automatic 
Train Operation) for many years. However, as the road traffic
is far more complex, self-driving cars had never been in 
actual use. Thanks to the development of Machine Learning 
and Computer Vision, self-driving cars does not seem to be 
impossible nowadays. Thus, we would like to design a 
self-driving assistant program in the computer simulation 
environment.

The best performed models were chosen for this project to 
achieve the best performance. 


\section{Pre Process}


\section{Process}

\subsection{Lane Line Detection}

\subsection{Obstacle Detection}


\section{Post Process}


\section{Distributed Scheduling}

\subsection{Containerization}
Containerization has been widely used in many industries. 
Compare to traditional virtual machine, the performance 
and resources loss are reduced to a large extent benefits 
from removing the guest OS and hardware virtualization. 
Docker, as the most popular container platform, are used here
in this project to containerlize our algorithm so that 


\subsection{Container Orchestration}

\section{Conclusions}


\clearpage

\bibliographystyle{splncs}
\bibliography{egbib}
\end{document}
