% last updated in April 2002 by Antje Endemann
% Based on CVPR 07 and LNCS, with modifications by DAF, AZ and elle, 2008 and AA, 2010, and CC, 2011; TT, 2014; AAS, 2016

\documentclass[runningheads]{llncs}
\usepackage{graphicx}
\usepackage{amsmath,amssymb} % define this before the line numbering.
\usepackage{ruler}
\usepackage{color}
\usepackage[width=122mm,left=12mm,paperwidth=146mm,height=193mm,top=12mm,paperheight=217mm]{geometry}
\begin{document}
% \renewcommand\thelinenumber{\color[rgb]{0.2,0.5,0.8}\normalfont\sffamily\scriptsize\arabic{linenumber}\color[rgb]{0,0,0}}
% \renewcommand\makeLineNumber {\hss\thelinenumber\ \hspace{6mm} \rlap{\hskip\textwidth\ \hspace{6.5mm}\thelinenumber}}
% \linenumbers
\pagestyle{headings}
\mainmatter
\def\ECCV18SubNumber{31}

\title{Self-Driving Assistant in Computer Simulation 
Environment}

\titlerunning{ENGN4528 Group \ECCV18SubNumber}

\authorrunning{ENGN4528 Group \ECCV18SubNumber}

\author{Zhiyuan Chen, Xingyuan Xu, Qiusi Xiang, Bisyri 
Hisham, Kavinenh Mohanraj}
\institute{Australian National University}


\maketitle

% FORMAT GUIDE
% No more than 60 characters in a single line [1][2]
% All lines must ends with a whitespace
% Every section ends with two empty lines
% Every subsection ends with a empty line
% [1]: Punctuation at the end of the line does not count
% [2]: Whitespace at the end of the line does not count

\begin{abstract}
This project implement a self-driving assistant in computer 
simulation environment with Spatial CNN and Mask RCNN on a 
distributed system.
\dots
\keywords Self-Driving, Spatial CNN, Mask RCNN, Distributed 
System
\end{abstract}


\section{Introduction}
Self-driving technologies had been used in airplanes (known 
as AP for Auto Pilot), and trains (known as ATO for 
Automatic Train Operation) for decades. However, as the 
road traffic is far more complex, self-driving cars have 
never been in commerical use. Thanks to the development of 
Machine Learning, Computer Vision and most importantly the 
hardwares, self-driving cars does not seem to be impossible 
today. Thus, we designed a self-driving assistant program 
in the computer simulation environment. 

The best performed models were chosen for this project to 
achieve the best performance. 

\section{Local Process}
\subsection{Screenshot}
Taking a screenshot can be harder than it looks. 
ImageGrab.grab() from PIL is a good approach. However, 
it costs 300-400ms to take a single screenshot which is 
obviously unacceptable. We test many different functions 
from different package, as a result, Python-MSS is much 
faster than any other package in the comparison with a 
average speed of 60fps, and was used at last.

\subsection{Encode \& Decode}
As we are using message queue as our message-oriented 
middleware, we must encode(serialize) the image prior to 
transimission and decode(deserialize) upon receive message. 
We tried three different ways to encode our image, JSON, 
pickle, and imencode. 

JSON is widely used in message queue, however, as ndarray 
used for image cannot be directly convert to JSON, we have 
to serialize the image first. Multiple attempts had been 
made, we succeed to encode image in milliseconds at last, 
unfortunately, we could not decode the image fast enough 
and we had to abandon JSON. 

Directly pack the image using pickle can be another method, 
we tried cPickle here instead for better performance. The 
encode and decode speed is the best among all methods, 
however, the size of the message is also several times 
larger than the size of the image. Messages of this size 
would place a heavy burden on the network, and was finally 
deprecated. 

The Joint Photographic Experts Group issued a synonym 
standard for still pictures compression in 1992, after 
27 years, the JPEG algorithm had became the most commonly 
used algorithm in image compression on this planet. OpenCV 
provides a built-in fucntion, i.e. imencode, which can 
convert a image from ndarray to byte stream within 
milliseconds. And more importantly, thanks to JPEG 
algorithm, the size of message is also ideal. 


\section{Worker Process}

\subsection{Lane Line Detection}

Spatial CNN was introduced by Pan in 2017. They 
demonstrated

\subsection{Obstacle Detection}


\section{Distributed System}

\subsection{Message-Oriented Middleware}
Message-Oriented Middleware or MOM was used to transmit 
data between each cluster and is the core of the whole 
distributed system. MOM deliver messages asynchronously 
which benefits us a alot. It allow us to remove 
dependencies between each algorithm and decouple our 
program. Moreover, it also increase the reliability of our 
program, since the system would continue to work even 
though one or more of the algorithms encountered fatal 
problems and unable to recover.

Message Queue was used in our program as the MOM. We 
compared many message queues, including kafka, RabbitMQ, 
NATS, and Redis to find the best performed message queue 
in latency. As Advanced Message Queueing Protocol 
implementation, kafka and RabbitMQ are a lot slower than 
Redis and NATS in 1KB and 5KB text. However, when it comes
to 1MB text, RabbitMQ and kafka are 200 times faster than 
NATS and Redis in 99.99th percentile. At last, we chose 
RabbitMQ as our MOM since it works slightly better in large 
message transmission. In addition, to improve the real-time 
capabilities, a Time To Live limit is set to 1 millisecond.


\subsection{Containerization}
Containerization has been widely used in many industries. 
Compare to traditional virtual machine, the performance 
and resources loss are reduced to a large extent benefits 
from removing the guest OS and hardware virtualization. 
Moreover, using container would also reduce the effort to 
deploy in future works. 

Docker, the most popular container platform, are used here 
in this project to containerize our algorithm. Compare to 
other containerization strategy, e.g. LXD, Docker focused 
on running a program instead of a whole system which make 
it easier to use. In addition, Docker also support multiple 
system which is more convinent for development. And most 
importantly the nvidia-docker make Docker more compatible 
with neural networks.

\subsection{Container Orchestration}


\section{Conclusions}


\clearpage

\bibliographystyle{splncs}
\bibliography{bib}
\end{document}
